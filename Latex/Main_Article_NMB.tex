\documentclass[]{article}
\usepackage[]{algorithm2e}
\usepackage{amsmath}
\usepackage{framed}
\usepackage[english]{babel}
\usepackage{blindtext}

%opening
\title{Practicum NMB : Eigenwaardenproblemen}
\author{Matthijs van Keirsblick en Harald Schäfer}
\date{vrijdag 25 april 2014}

\newcommand{\opgave}[1]{\section*{Opgave #1}}


\begin{document}

'\blindtext'

\maketitle
\opgave{1}

We beginnen door een QR-factorisatie te berekenen van $A_{0}$ met eender welke methode. Vervolgens berekenen we $b = Q^{*}*b_{0}$. Met deze waarden kunnen we de iteratieve berekening starten die hieronder beschreven staat. De $G_{x}$ matrices zijn givens transformaties om de toegevoegde rijen van K terug op nul te stellen om zo terug een bovendriehoeksmatrix R te bekomen waarin de nieuwe waarden in verwerkt zijn.

\begin{framed}

\begin{algorithm}[H] 
 $Q^{(0)}*R^{(0)} = A_{0}$\\
 $b = Q^{(0)*}*b_{0}$\\
 \For{i = 1 to k}{
  $f = n*d$\\
  $R^{(i)} = G_{f}^{*}*...*G_{1}^{*}*
  \begin{bmatrix}
    	R^{(i-1)}	\\
    	K
    \end{bmatrix}$\\
  $R^{(i)} = R^{(i)}(:n,:)$\\
  $Q^{(i)} = I*G_{1}*...*G_{f} $\\
  $b^{(i)} = Q^{(i)*}*
  \begin{bmatrix}
      	b^{(i-1)}	\\
      	c
      \end{bmatrix}$\\
  $b^{(i)} = b^{(i)}(:n,:)$\\
 }
 
 
% \caption{How to compute incremental least squares solution with QR factorisation}
\end{algorithm}

\end{framed}
Aan het einde van elke iteratie is het mogelijk om $x^{(i)}$ te berekenen door achterwaardse substitutie toe te passen op de vergelijking $R^{(i)}*x^{(i)} = b^{(i)}$. Omdat na elke iteratie maar $R^{(i)}$ en  $b^{(i)}$ opgeslagen moet worden is het duidelijk dat het gebruikte geheugen niet toeneemt. Omdat de grootte van de matrices R en b niet toeneemt neemt het rekenwerk ook niet toe met elke iteratie. Voor het berekenen van een Givens-transformatie zijn 2 delingen, 2 vermenigvuldigen, een optelling en een vierkwantswortel nodig. Er moeten n*d Givens-transformaties berekend worden per iteratie. Een vermenigvuldiging met een rotatiematrix van grootte n+d zoals in dit geval vraagt 4*(n+d) vermenigvuldigingen en 2*(n+d) optellingen. Er gebeuren 2*n*d -1 van die matrix vermenigvuldiginge per iteratie. Tot slot  gebeurt er nog voor de berekening van de $b^{(i)}$ een matrix vermenigvuldiging waarvoor $(n+d)^2$ vermenigvuldigingen gebeuren en $(n+d-1)^2$ optellingen.

\begin{itemize}
  \item $n*d*2$ delingen
  \item $n*d$ vierkantswortels
  \item $n*d*2 + 2*(n+d)*(2*n*d-1) + (n+d-1)^2 \approx 4*n^2*d$ optellingen
  \item $n*d*4 + 4*(n+d)*(2*n*d-1) + (n+d)^2 \approx 8*n^2*d$ vermenigvuldigingen
\end{itemize}

%\section{}

\end{document}
